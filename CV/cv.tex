%%%%%%%%%%%%%%%%%
% This is an sample CV template created using altacv.cls
% (v1.6.5, 3 Nov 2022) written by LianTze Lim (liantze@gmail.com). Compiles with pdfLaTeX, XeLaTeX and LuaLaTeX.
%
%% It may be distributed and/or modified under the
%% conditions of the LaTeX Project Public License, either version 1.3
%% of this license or (at your option) any later version.
%% The latest version of this license is in
%%    http://www.latex-project.org/lppl.txt
%% and version 1.3 or later is part of all distributions of LaTeX
%% version 2003/12/01 or later.
%%%%%%%%%%%%%%%%

%% Use the "normalphoto" option if you want a normal photo instead of cropped to a circle
% \documentclass[10pt,a4paper,normalphoto]{altacv}

\documentclass[10pt,a4paper,ragged2e,withhyper]{altacv}
%% AltaCV uses the fontawesome5 and packages.
%% See http://texdoc.net/pkg/fontawesome5 for full list of symbols.

% Change the page layout if you need to
\geometry{left=1.25cm,right=1.25cm,top=1.5cm,bottom=1.5cm,columnsep=1.2cm}

% The paracol package lets you typeset columns of text in parallel
\usepackage{paracol}

% Change the font if you want to, depending on whether
% you're using pdflatex or xelatex/lualatex
\ifxetexorluatex
  % If using xelatex or lualatex:
  \setmainfont{Roboto Slab}
  \setsansfont{Lato}
  \renewcommand{\familydefault}{\sfdefault}
\else
  % If using pdflatex:
  \usepackage[rm]{roboto}
  \usepackage[defaultsans]{lato}
  % \usepackage{sourcesanspro}
  \renewcommand{\familydefault}{\sfdefault}
\fi

% Change the colours if you want to
\definecolor{SlateGrey}{HTML}{2E2E2E}
\definecolor{LightGrey}{HTML}{666666}
\definecolor{DarkPastelRed}{HTML}{450808}
\definecolor{PastelRed}{HTML}{8F0D0D}
\definecolor{GoldenEarth}{HTML}{E7D192}
\colorlet{name}{black}
\colorlet{tagline}{PastelRed}
\colorlet{heading}{DarkPastelRed}
\colorlet{headingrule}{GoldenEarth}
\colorlet{subheading}{PastelRed}
\colorlet{accent}{PastelRed}
\colorlet{emphasis}{SlateGrey}
\colorlet{body}{LightGrey}

% Change some fonts, if necessary
\renewcommand{\namefont}{\Huge\rmfamily\bfseries}
\renewcommand{\personalinfofont}{\footnotesize}
\renewcommand{\cvsectionfont}{\LARGE\rmfamily\bfseries}
\renewcommand{\cvsubsectionfont}{\large\bfseries}


% Change the bullets for itemize and rating marker
% for \cvskill if you want to
\renewcommand{\itemmarker}{{\small\textbullet}}
\renewcommand{\ratingmarker}{\faCircle}

%% Use (and optionally edit if necessary) this .tex if you
%% want to use an author-year reference style like APA(6)
%% for your publication list
% \input{pubs-authoryear}

%% Use (and optionally edit if necessary) this .tex if you
%% want an originally numerical reference style like IEEE
%% for your publication list
%\input{pubs-num}

%% sample.bib contains your publications
%\addbibresource{sample.bib}

\begin{document}
\name{Vincent Belpaire}
\tagline{Student at Ghent University}
%% You can add multiple photos on the left or right
%\photoR{2.8cm}{Globe_High}
% \photoL{2.5cm}{Yacht_High,Suitcase_High}

\personalinfo{%
  % Not all of these are required!
  \email{vincent.belpaire@gmail.com}
  \phone{0479-10-39-65}
  \mailaddress{Address, Durmestraat 4, 9140 Tielrode}
  \location{Location, Belgium}
  %\homepage{www.homepage.com}
  %\twitter{@twitterhandle}
  \linkedin{https://www.linkedin.com/in/vincent-belpaire-60941519a}
  \github{https://github.com/vbelpaire}
  %\orcid{0000-0000-0000-0000}
  %% You can add your own arbitrary detail with
  %% \printinfo{symbol}{detail}[optional hyperlink prefix]
  % \printinfo{\faPaw}{Hey ho!}[https://example.com/]
  %% Or you can declare your own field with
  %% \NewInfoFiled{fieldname}{symbol}[optional hyperlink prefix] and use it:
  % \NewInfoField{gitlab}{\faGitlab}[https://gitlab.com/]
  % \gitlab{your_id}
  %%
  %% For services and platforms like Mastodon where there isn't a
  %% straightforward relation between the user ID/nickname and the hyperlink,
  %% you can use \printinfo directly e.g.
  % \printinfo{\faMastodon}{@username@instace}[https://instance.url/@username]
  %% But if you absolutely want to create new dedicated info fields for
  %% such platforms, then use \NewInfoField* with a star:
  % \NewInfoField*{mastodon}{\faMastodon}
  %% then you can use \mastodon, with TWO arguments where the 2nd argument is
  %% the full hyperlink.
  % \mastodon{@username@instance}{https://instance.url/@username}
}

\makecvheader
%% Depending on your tastes, you may want to make fonts of itemize environments slightly smaller
% \AtBeginEnvironment{itemize}{\small}

%% Set the left/right column width ratio to 6:4.
\columnratio{0.6}

% Start a 2-column paracol. Both the left and right columns will automatically
% break across pages if things get too long.
\begin{paracol}{2}
\cvsection{Experience}

\cvevent{Technical service, signalisation}{Stad Sint-Niklaas}{July 2019 -- August 2019}{9100 Sint-Niklaas, België}
\begin{itemize}
\item Placement and repair of traffic signs
\end{itemize}

\divider

\cvevent{Administration, communication service}{Stad Sint-Niklaas}{August 2019}{9100 Sint-Niklaas, België}
\begin{itemize}
\item Preparation of letters
\item Communication line for complaints and problems
\end{itemize}

\divider

\cvevent{Programmer}{ST engineering}{August 2021 + August 2022}{9100 Sint-Niklaas, België}
\begin{itemize}
\item Workflow automation and enhancement
\end{itemize}

\divider

\cvevent{Committee on Constitutional Affairs I}{European Youth Parlement}{26/02/2019 - 01/03/2019}{Liége}
Put into a group of strangers, we get the task to tackle a problem concerning Europe and its people: "How does Europe gain connection and trust from its citizens?".
After one week, while the strangers turned into colleagues and friends, we had to write a resolution with possible solutions and defend it in a debate with others.

\divider

\cvevent{Predictive policing}{University of Ghent and Stad Gent}{2021/09-2021/06}{}
One of the most common complaints in daily life concerns illegal dumping, according to a poll organized by Ghent around 2019. 
In a group of four students, myself included, we were given the task to find patterns in how illegal dumping is distributed spatially and over time and to design
a predictive model that could help in preventing illegal dumping.

\medskip

\cvsection{A Day of My Life}

% Adapted from @Jake's answer from http://tex.stackexchange.com/a/82729/226
% \wheelchart{outer radius}{inner radius}{
% comma-separated list of value/text width/color/detail}
\wheelchart{1.5cm}{0.5cm}{%
  6/8em/accent!30/Sleep to regain stamina,
  3/8em/accent!40/Reading books,
  8/8em/accent!60/Study and enhance skills,
  2/10em/accent/Sports,
  5/6em/accent!20/Spending time with family and friends
}

% use ONLY \newpage if you want to force a page break for
% ONLY the current column
\newpage

%\cvsection{Publications}

%% Specify your last name(s) and first name(s) as given in the .bib to automatically bold your own name in the publications list. 
%% One caveat: You need to write \bibnamedelima where there's a space in your name for this to work properly; or write \bibnamedelimi if you use initials in the .bib
%% You can specify multiple names, especially if you have changed your name or if you need to highlight multiple authors. 
%\mynames{Lim/Lian\bibnamedelima Tze,
%  Wong/Lian\bibnamedelima Tze,
%  Lim/Tracy,
%  Lim/L.\bibnamedelimi T.}
%% MAKE SURE THERE IS NO SPACE AFTER THE FINAL NAME IN YOUR \mynames LIST

%\nocite{*}

%\printbibliography[heading=pubtype,title={\printinfo{\faBook}{Books}},type=book]

%\divider

%\printbibliography[heading=pubtype,title={\printinfo{\faFile*[regular]}{Journal Articles}},type=article]

%\divider

%\printbibliography[heading=pubtype,title={\printinfo{\faUsers}{Conference Proceedings}},type=inproceedings]

%% Switch to the right column. This will now automatically move to the second
%% page if the content is too long.
\switchcolumn

\cvsection{My Life Philosophy}

\begin{quote}
``Nothing exists purely on itself, but only in relation to something else.''
\end{quote}

\cvsection{Most Proud of}

\cvachievement{\faHeartbeat}{Loyalty and independence}{Despite a dramatical incident I stay loyal to the ones I care for and strive for their's and mine independence.}

\divider

\cvachievement{\faTrophy}{Ambitious}{Instead of taking the "easy" road I chose to follow my interests, even if it makes it harder.}

%\divider

%\cvachievement{\faHeartbeat}{}{more details about it of course}

\cvsection{Strengths}

\cvtag{Hard-working}
\cvtag{Flexible}\\
\cvtag{Motivator \& Teacher}

\divider\smallskip

\cvtag{C++}
\cvtag{Python}\\
\cvtag{Linux}
\cvtag{Latex}

\cvsection{Languages}

\cvskill{English}{4}
\divider

\cvskill{Dutch}{4}

%% Yeah I didn't spend too much time making all the
%% spacing consistent... sorry. Use \smallskip, \medskip,
%% \bigskip, \vspace etc to make adjustments.
\medskip

\cvsection{Education}

\cvevent{General secondary education}{Broedersschool Sint-Niklaas}{Sept 2013 -- June 2019}{}

\divider

\cvevent{B.Sc.\ in Biomedical Engineering}{Ghent University}{Sept 2020 -- Ongoing}{}

\divider

\cvevent{B.Sc.\ in Physics and Astronomy}{Ghent University}{Sept 2022 -- Ongoing}{}

% \divider

%\cvsection{Referees}

% \cvref{name}{email}{mailing address}
%\cvref{Prof.\ Alpha Beta}{Institute}{a.beta@university.edu}
%{Address Line 1\\Address line 2}

%\divider

%\cvref{Prof.\ Gamma Delta}{Institute}{g.delta@university.edu}
%{Address Line 1\\Address line 2}


\end{paracol}


\end{document}
